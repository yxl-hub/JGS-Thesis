\documentclass[12pt,a4paper]{article}

% 页边距设置
\usepackage[a4paper,top=2.5cm,bottom=2.5cm,right=2.5cm,left=3.2cm]{geometry}

% Times New Roman 字体
\usepackage{fontspec}
\setmainfont{Times New Roman}

% 行距与段落间距
\usepackage{setspace}
\setstretch{1.5} % 行距 1.5 倍
\setlength{\parindent}{0pt} % 无首行缩进
\setlength{\parskip}{1em}   % 段落间距

% 标题格式
\usepackage{titlesec}
\titleformat{\section}{\fontsize{14pt}{14pt}\bfseries}{\thesection}{1em}{}
\titleformat{\subsection}{\fontsize{14pt}{14pt}\bfseries}{\thesubsection}{1em}{}
\titleformat{\subsubsection}{\fontsize{12pt}{12pt}\bfseries}{\thesubsubsection}{1em}{}

% 设置 \section 标题的间距
\titlespacing*{\section}
    {0pt}                % 1. 左边距 (left margin)
    {1\baselineskip}     % 2. 段前间距 (space before)
    {0pt}                % 3. 段后间距 (space after)

% 设置 \subsection 标题的间距
\titlespacing*{\subsection}
    {0pt}                % 1. 左边距 (left margin)
    {1\baselineskip}     % 2. 段前间距 (space before)
    {0pt}                % 3. 段后间距 (space after)

% 页码居中
\usepackage{fancyhdr}
\pagestyle{fancy}
\fancyhf{}
\renewcommand{\headrulewidth}{0pt} % 移除页眉线
\fancyfoot[C]{\thepage}

% APA 7 参考文献
\usepackage[style=apa,sortcites=true,sorting=nyt,backend=biber]{biblatex}
\DeclareLanguageMapping{english}{english-apa}
\addbibresource{EE2.bib} % 参考文献文件

\usepackage{tocloft}
% 定义 section 条目的引导线为虚线点
\renewcommand{\cftsecleader}{\cftdotfill{\cftdotsep}}
% 定义 subsection 条目的引导线为虚线点
\renewcommand{\cftsubsecleader}{\cftdotfill{\cftdotsep}}

%图片及表格
\usepackage{graphicx} % 用于插入图表
\usepackage{booktabs} % 用于漂亮的表格线
\usepackage{multirow} % 如果需要合并行
\usepackage{tabularx} % 用于自动调整列宽
\usepackage{float}    % 支持表格/图片的[H]强制固定位置参数

%公式及超链接
\usepackage{amsmath}
\usepackage{hyperref}
\hypersetup{
    colorlinks=true, % 设置链接颜色
    linkcolor=black, % 目录链接颜色
    urlcolor=cyan,   % URL链接颜色
    citecolor=black  % 引用链接颜色
}

% 封面页命令
\newcommand{\maketitlepage}[3]{%
    \thispagestyle{empty}
    \begin{center}
        \vspace*{1cm}
        {\fontsize{24pt}{24pt}\bfseries #1\par}
        \vspace{12cm}
        {\fontsize{20pt}{20pt}\bfseries #2\par}
        \vspace{1cm}
        {\fontsize{20pt}{20pt}\bfseries #3\par}
    \end{center}
    \newpage
}

\begin{document}

% 封面
\maketitlepage{REPowerEU: An Imperfect Yet Effective Response 
to Energy Security Issues Arising from the Russia-Ukraine War}
{Yu Xilong}{Pre-sessional EAP Course \\ 2025.6.6}

\pagenumbering{roman} % 从这里开始使用罗马数字页码 (i, ii, iii...)
% 目录
\tableofcontents
\clearpage

\pagenumbering{arabic} % 从这里开始使用阿拉伯数字页码 (1, 2, 3...)
% 摘要
\section*{Abstract}
This essay evaluates the effectiveness of the EU's REPowerEU Plan, launched in 2022 to reduce dependence on Russian energy and enhance energy security following the invasion of Ukraine.
Analyzing three years of implementation data, the study finds the policy largely successful in its core short-term objectives. Key achievements include exceeding the gas demand reduction target (17\%),
drastically reducing Russian fossil fuel imports (e.g., gas down to 19\% from 45\%), and accelerating renewable energy deployment, surpassing targets. Diversification mechanisms like the EU Energy 
Platform facilitated significant new supply contracts. However, the analysis identifies significant long-term challenges, particularly divergent interests among member states regarding the energy 
transition phase (e.g., coal dependency in Eastern Europe, nuclear debates) and unresolved concerns about distributional equity within the Plan's framework. While effectively bolstering immediate 
energy security, the essay concludes that REPowerEU requires more flexible, country-specific implementation strategies to address internal disparities for sustained success.
\vspace{1\baselineskip} % 3 行间距


% 正文
\section{Introduction}
The Russia-Ukraine War (RUW) began in February 2022 and continues to this day. In response to Russia's aggression against Ukraine, the European Union implemented a total of nine rounds of sanctions in 
2022, including financial sanctions, energy sanctions, and others. Russia, however, is a significant supplier of oil and natural gas to the EU. The above sanctions have greatly impacted the stability 
of the EU energy market, leading to annual energy inflation surged by 38.3\%, with natural gas prices skyrocketing by 52.2\% (Sokhanvar Lee, 2023). To reduce reliance on Russian energy and ensure 
the energy security of the EU, the EU has launched the REPowerEU Plan. The plan initiative has now been in effect for three years, yet a comprehensive analysis of its efficacy remains absent. This 
essay, therefore, undertakes an evaluation of the policy's effectiveness and argues that while the policy demonstrates significant efficacy, it is not perfect.

This essay will first outline the policy core measures and objectives, analyze its achievements based on data, and subsequently identify the challenges and controversies it faces.

\section{Core Measures and Objectives of REPowerEU}
The REPowerEU initiative outlines measures in three key areas to achieve the core objectives of reducing dependence on Russian energy and ensuring energy security: energy conservation and efficiency 
improvement, diversification of energy supply, and accelerated deployment of renewable energy sources(European Commission, 2022). Regarding the definition of energy security, there is currently no 
unified consensus. Diverse definitions primarily emphasize Energy availability, infrastructure, energy prices, societal effects, environment, governance, and energy efficiency(Ang et al., 2015). This 
essay adopts the definition provided by the IEA: "Reliable, affordable access to all fuels and energy sources."(IEA, 2025)

REPowerEU aims to achieve its objectives through a combination of increasing supply, reducing demand, and facilitating transformation. To enhance supply, it promotes diversification of energy sources, 
with specific measures including the establishment of the "EU Energy Platform" and the "AggregateEU" mechanism, which have successfully matched nearly 100 billion cubic meters of natural gas supply and 
demand, accounting for 30\% of the EU's total needs. Additionally, it pushes for an increase in liquefied natural gas (LNG) import capacity, aiming for an import capability of 50 billion cubic meters by 
2024, projected to rise to 70 billion cubic meters by the end of that year(European Commission, 2022). With regard to demand reduction, the EU has established a voluntary target of cutting gas 
consumption by 15\% compared to the 2017-2021 average through the "Regulation on Reducing Gas Demand." It has also revised the "Energy Efficiency Directive," mandating a 11.7\% reduction in final energy 
consumption by 2030 compared to the 2020 reference scenario, with a focus on promoting building renovation (accounting for 52\% of EU gas consumption) and industrial electrification (European Commission, 
2022). To achieve transition goals, the EU has mandated the installation of rooftop photovoltaic systems in new public buildings starting from 2025 and in residential buildings from 2029. Additionally, 
the EU plans to produce and import 10 million tons of green hydrogen each by 2030, while establishing three major import corridors(European Commission, 2022).

\section{Conclusion}
Facing the energy crisis precipitated by the Russia-Ukraine war, the European Union swiftly formulated the REPowerEU initiative to address the situation. This plan aims to wean the EU off its reliance 
on Russian energy and accelerate the energy transition to ensure long-term energy security. Over the three years of its implementation, the initiative has significantly reduced energy imports from 
Russia and either met or exceeded the targets for energy transition. Consequently, it has successfully achieved its initial objectives: decoupling from Russian energy dependence and safeguarding EU 
energy security. In other words, the EU has, up to this point, achieved the short-term objectives set out in the REPowerEU Plan, but there are still issues regarding the long-term goals associated 
with the energy transition. Such as the disparate interests among EU member states during the energy transition, leading to notable variations in the progress of energy transformation. Therefore, 
this essay contends that while the plan effectively ensures EU energy security, it is not without flaws.

Based on the aforementioned analysis, this essay posits that the EU should establish more precise objectives and conditions for each REPowerEU measure, allowing individual countries to implement the 
plan flexibly based on their specific circumstances. (1488 Words)

\section{Test}

% 调整特定章节的间距
\titlespacing{\subsection}{0pt}{0\baselineskip}{0pt} % 临时调整间距,左缩进,上间距,下间距

\subsection{Test for references}
This is the first reference\parencite{andrikogiannopoulou_reassessing_2019}. This reference is intended to test the simultaneous reference of two articles\parencite{barras_false_nodate,lee_boosting_2024}.

\textcite{wang_robust_2024} deliver a robust theory.

% 调整特定章节的间距
\titlespacing{\subsection}{0pt}{1\baselineskip}{0pt} % 调整回原始间距,左缩进,上间距,下间距

\subsection{Test for image and table}

This is the first table % 用的是自带的tabular*宏包,用于自动对齐文本宽度(而宏包tabular仅根据内容调整表格宽度)。内容无需换行或内容短优先使用

\begin{table}[H]
\centering
\caption{a table}
\label{tab:tab 1}
\begin{tabular*}{\textwidth}{@{\extracolsep{\fill}} lcr @{}} % lcr代表三列,分别是左对齐、居中、右对齐,前面的fill用于自动调整列间距(以空白填充)
\toprule
Item & Quantity & Price \\
\midrule
Apple & 10 & \$1.25 \\
Banana & 5 & \$0.75 \\
Orange & 8 & \$1.50 \\
\bottomrule
\end{tabular*}
\end{table}

This is the second table % 用的是tabularx宏包,用于自动调整列宽。存在长内容优先使用

\begin{table}[htbp] % 这里的htbp表示浮动位置,h=here, t=top, b=bottom, p=page of floats
\centering
\caption{a more complex specification table}
\label{tab:tab 2}
\begin{tabularx}{\textwidth}{@{}llX@{}} % llX代表三列,前两列左对齐,第三列自动调整宽度以填满表格宽度
\toprule
Category & Item & \multicolumn{1}{c}{Description} \\
\midrule
\multirow{2}{*}{Fruit} % Fruit跨两行,使用multirow宏包实现
& Apple & A common temperate fruit, rich in vitamins and fiber. \\
& Orange & A citrus fruit known for its high vitamin C content. \\
\midrule
\multicolumn{2}{@{}l}{Total} & Two types of fruit \\
\bottomrule
\end{tabularx}
\end{table}

This is an image
\begin{figure}[htbp] % h-here, t-top, b-bottom, p-page
\centering % 居中显示图片
\includegraphics[width=0.8\textwidth]{../pic/EE2_1.png} % 插入图片,宽度为文本宽度的80%
\caption{Stranger Things}
\label{fig:fig 1} % 设置引用标签
\end{figure}

\subsection{Test for formula}
% 无编号或行内公式, 用$...$或\[...\], 建议使用后者,但我喜欢用前者,因为MarkDown也是用的前者
This is an inline formula: $E=mc^2$.

% 有编号并且可以交叉引用
\begin{equation}
\label{eq:equ 1}
e_i^{\text{agg}*} = \frac{1}{n_i} \sum_{k=1}^{n_i} \binom{n_i}{k}^{-1} \sum\limits_{\mathcal{S}_{ki} \in \mathcal{B}_{ki}} \pi^k A_{i,\mathcal{S}_{ki}}
\end{equation}

% 参考文献
\newpage
\printbibliography[title={References}]

\end{document}
